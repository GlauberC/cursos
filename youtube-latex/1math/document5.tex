\documentclass[12pt, a4paper]{article}	%Cria um artigo com tamanho 12pt papel a4 e em 2 colunas

\usepackage[brazilian]{babel}	%Traduz documento para português do brasil
\usepackage[utf8]{inputenc}
\usepackage{lipsum}		%adiciona o pacote lipsum
\title{Este é o título do meu artigo}	%adiciona titulo ao trabalho
\author{Glauber Carvalho\thanks{Agradeço a minha mãe}}	%Adiciona nome do autor e agradecimentos

\date{\today}	%coloca a data de hoje no seu trabalho

\newtheorem{teorema}{Teorema}[section]	% Segue as mesmas definições da sessão
\newtheorem{definicao}[teorema]{Definição} % Segue as mesmas definições dos teoremas
\newtheorem{proposicao}{Proposição}[section] % Segue as mesmas definições da sessão mas diferente do teorema e da definição

\begin{document}
	\section{Primeira Parte}
	\begin{teorema}
		Todo quadrado tem quatro ângulos retos
	\end{teorema}
	\begin{teorema}
		Todo triângulo tem três lados
	\end{teorema}
	\begin{definicao}
		Um quadrado é um poligono de quatro lados.
	\end{definicao}
	\begin{proposicao}[Pentágono] % Apelido
		Um pentagono tem cinco lados
	\end{proposicao}


	\section{Segunda Parte}
	\begin{teorema}
		Todo quadrado tem quatro ângulos retos
	\end{teorema}
	\begin{teorema}
		Todo triângulo tem três lados
	\end{teorema}
	\begin{definicao}
	Um quadrado é um poligono de quatro lados.
	\end{definicao}
	
		
\end{document}