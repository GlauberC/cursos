\documentclass[12pt, a4paper]{article}	%Cria um artigo com tamanho 12pt papel a4 e em 2 colunas

\usepackage[brazilian]{babel}	%Traduz documento para português do brasil
\usepackage{lipsum}		%adiciona o pacote lipsum
\title{Este é o título do meu artigo}	%adiciona titulo ao trabalho
\author{Glauber Carvalho\thanks{Agradeço a minha mãe}}	%Adiciona nome do autor e agradecimentos

\date{\today}	%coloca a data de hoje no seu trabalho

\begin{document}
	\setlength{\parindent}{2cm}
	\lipsum[1]
	\noindent\lipsum[1]
	\newpage
	\lipsum[2]
	\vspace{2cm}
	Listas de compras: Batata\hspace{1cm}, Feijão\hspace{1cm}, Arroz
	
	Esta linha deve ser quebrada aqui\hfill\break a partir desse ponto começa outra linha
	
	Esta linha deve ser quebrada aqui\newline a partir desse ponto começa outra linha
	\\\\
	Palavra em \textbf{Negrito}, em \textit{itálico} , com \emph{enfase} e \underline{sublinhada}
	
		
\end{document}