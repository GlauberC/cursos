\documentclass[12pt, a4paper, twocolumn]{article}	%Cria um artigo com tamanho 12pt papel a4 e em 2 colunas

\usepackage[brazilian]{babel}	%Traduz documento para português do brasil
\usepackage{lipsum}		%adiciona o pacote lipsum
\title{Este é o título do meu artigo}	%adiciona titulo ao trabalho
\author{Glauber Carvalho\thanks{Agradeço a minha mãe}}	%Adiciona nome do autor e agradecimentos

\date{\today}	%coloca a data de hoje no seu trabalho

\begin{document}
	\begin{titlepage}
		\maketitle
	\end{titlepage}
	
	\begin{abstract}
		\lipsum[1]
	\end{abstract}

	\section{Primeiro Tema}
	
	
		Este é o texto do meu documento 	
		
		2º paragrafo começa aqui depois de dois enters\\ 	
		3º paragrafo começa aqui depois de duas barras após o segundo paragrafo\newline
		4º paragrafo começa aqui depois de uma barra newline depois do terceiro paragrafo\par
		5º paragrafo começa aqui depois de uma barra par depois do quarto paragrafo
		
		\begin{center}
			Texto centralizado
		\end{center}
	
		\begin{flushleft}
			Alinhamento à esquerda
		\end{flushleft}
		
		\begin{flushright}
			Alinhamento à direita
		\end{flushright}
		
\end{document}