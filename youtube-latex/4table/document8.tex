\documentclass[12pt, a4paperm, openany]{article}	%openany remove a pagina posterior ao índice
\usepackage[brazilian]{babel}	%Traduz documento para português do brasil
\usepackage[utf8]{inputenc}
\usepackage{lipsum}		%adiciona o pacote lipsum
\title{Este é o título do meu artigo}	%adiciona titulo ao trabalho
\author{Glauber Carvalho\thanks{Agradeço a minha mãe}}	%Adiciona nome do autor e agradecimentos

\date{\today}	%coloca a data de hoje no seu trabalho



\begin{document}
	\begin{tabular}{|l||r|c|c|}	%coluna alinhada esq dir centro centro
		\hline
		\multicolumn{2}{|c|}{Dados 1}	&	\multicolumn{2}{|c|}{Dados 2}	\\
		\hline
		Jogador	&	Posição			&	Gols Marcados	&	Faltas	\\
		\hline
		\hline
		João	&	Atacante		&	5				&	2		\\
		\hline
		José	&	Lateral			&	1				&	1		\\
		\hline
		Mário	&	CentroAvante	&	4				&	1		\\
		\hline
	\end{tabular}
	\\ 
	\\

	\begin{tabular}{|c|c|}
		\hline
		Alimentos & Nota \\ \hline
		 Laranja  &  10  \\ \hline
		   Uva    &  10  \\ \hline
		  Pinha   &  7   \\ \hline
	\end{tabular}
	\\
	\\
	
	\begin{tabular}{lr@{.}l}	% Não há 3 colunas, mas sim um alinhamento com ponto
		Expressão & \multicolumn{2}{c}{Valor} \\ \hline
		$\pi$ & 3 & 1416 \\
		$\pi^\pi$ & 36 & 46 \\
		$(\pi^\pi)^\pi$ & 80662 & 7 \\
		\hline
	\end{tabular}

		
\end{document}